\documentclass[12pt,letterpaper,notitlepage,onecolumn,final,openbib]{article}
\usepackage[latin1]{inputenc}
\usepackage{amsmath}
\usepackage{amsfonts}
\usepackage{amssymb}
\usepackage[T1]{fontenc}
\usepackage{graphicx}
\usepackage{dcolumn}
\usepackage{rotating}
\usepackage{setspace}
\usepackage{epigraph}
\usepackage{url}
\usepackage{chicago}
\usepackage{booktabs}
\author{Martin Lefebvre}
\title{Comps}
\begin{document} 
	
\section{Question 1}
\begin{quotation}
	\textbf{Why is location an important factor in the geography of finance?}
\end{quotation}



\section{Question 2}
\begin{quotation}
	\textbf{Are there circumstances in which agglomeration economies advantage might be overstated or perhaps even illusionary? In answering, provide at least one example.}
\end{quotation}

Agglomeration economies consists of processes that take advantage of economies of scale and network effects.  Furthermore, these processes operate on two important levels, that of the individual firm and that of the region.  However, it would be fallacious to believe that the net effect of agglomeration is always beneficial or even tangible.   

On the level of the firm, 
For example, many studies have shown that banks have diminishing returns on Capital when they pass 25 billion dollars in yearly revenue.  \cite{Yves2010_Myths}
%

%Firm  
%Too big to govern effectively
%	Banks - 
%Diseconomies of scale
%	Finance - waste of resources/locus for rent extraction
%	%	Q: You’re concerned that the financial sector is too big. Why?
%	%	A; The job of finance is to provide capital to companies. We do it to the tune of $250 billion a year in IPOs and secondary offerings. What else do we do? We encourage investors to trade about $32 trillion a year. So the way I calculate it, 99% of what we do in this industry is people trading with one another, with a gain only to the middleman. It’s a waste of resource
%
%	Competition between firms
%		Margin
%			Increasing completion should reduce profitability - all other assumptions being held 
%		Too Dominant
%		Labour
%			Too much competition for skilled labour may raise wages above profitabilty vs labour in another location
%	
%-Negative Network Effects
%	Congestion
%		Overload
%			Chicago Rail for logistics operations
%			Traffic Jam at rush hour
%	Pollution
%		Concentration of pollution 
%			Non-linear effects
%			

\section{Question 3}
\begin{quotation}
	\textbf{This question has the following parts: (1) Describe, explain and discuss ``Spatialization'' based on Fabrikant, Skupin et al.; (2) Next, provide a hypothetical for now, but as good-as-you-can-make-it, example of Spatialization using	some of the data for your study. Which of your data variables (pp. 14-15 in your Proposal) can be used? Pay attention to the X and Y axis, ensuring they make sense and have a clear label. State what these labels are; (3) What Distance Modeling could be applied to your example? (Cf. Chapter 6 on Distance Relationships by Chrisman); (4) Given `Investment DistanceSheds' generated in (3) in non-spatial `financial spaces', what Surface Modeling can be applied? (Cf. Chapter 8 on Spatial Analysis using Continuous Fields by Burrough et al); (5) What results can be expected from using this type of approach? Comment on this three-part `Surface Analysis of Distance Models derived from Spatialization of Investment Data' approach.}
\end{quotation}

\subsection{}
\subsection{}
\subsection{}
\subsection{}
\subsection{}


	\pagebreak
	\bibliographystyle{chicago}
	\bibliography{Mybibliography}
\end{document}