\documentclass[12pt,letterpaper,notitlepage,onecolumn,final,openbib]{article}
\usepackage[latin1]{inputenc}
\usepackage{amsmath}
\usepackage{amsfonts}
\usepackage{amssymb}
\usepackage[T1]{fontenc}
\usepackage{graphicx}
\usepackage{dcolumn}
\usepackage{rotating}
\usepackage{setspace}
\usepackage{epigraph}
\usepackage{url}
\usepackage{chicago}
\usepackage{booktabs}
\author{Martin Lefebvre}
\title{Comps}
\begin{document} 
	
\section{Question 1}
\begin{quotation}
	\textbf{Why is location an important factor in the geography of finance?}
\end{quotation}




\section{Question 2}
\begin{quotation}
	\textbf{Are there circumstances in which agglomeration economies advantage might be overstated or perhaps even illusionary? In answering, provide at least one example.}
\end{quotation}

Agglomeration economies consists of processes that take advantage of economies of scale and network effects.  Furthermore, these processes operate on two important levels, that of the individual firm and that of the region.  However, it would be fallacious to believe that the net effect of agglomeration are always tangible or beneficial.   

Economies of scale are frequently at the forefront of economics literature \cite{Panzar1977} with regards to firm-level competitiveness.  It should be noted that when drawn schematically in an idealized model, the cost curve is u shaped: costs decrease as production increases and once optimal conditions are surpassed, costs increase.  Financial commentatiors such Yves Smith have stipulated that despite the perennial promesses of synergy driven by the elimination of duplication of effort during bank mergers, the empirical evidence shows that the costs of managing large banks have diminishing returns when banks pass 25 billion dollars in yearly revenue \cite{Smith2010}.  


As it currently stands, there is no formal model for the spreading of tacit knowledge in an economic cluster.  While it might be a true that knowledge spreads via the ``old boy's network'' during time away from work and between employers, much of this literature pertains to the experience of information technology professionals located in California's Silicone Valley as well as the film industry in Hollywood.  This popular phenomenon might not be true in different cultural backgrounds such as the Cambridge information technology cluster, where there is very little off-the-clock fraternization between employees that have moved on to other employers due to long work hours and the social climate is more restrained, and thus there is less `shop-talk' and thus transmission of best practises between employers.       

On a regional scale, it is possible for agglomeration clusters to adversely affect broader economic conditions even though the clustering activity is profitable for the firms in the cluster.  For example, \citeN{Cecchetti2012} observes that the size of a country's financial sector is correlated with GDP growth only up to a point, and that over-sized financial sectors are a drag on economic growth.  This finding is explored in further detail in \citeN{Cecchetti2015}, and this working paper finds that large finance industries crowed-out other sectors of the economy, and this crowding out offers a mechanism for explaining the lower overall economic growth seen in the 2013 paper. 

Similarly, agglomerations can loose their competitive edge with their competition when scarcity of skilled labour raises wages above the level of competitiveness.  Furthermore, \citeN{Huber2012} finds that this phenomenon is especially hurtful to smaller companies in a cluster, for they have less resources to offer higher compensation when compared to larger firms.  

How to we tally the cost of environmental degradation?  A classic example is the UK's industrial midlands and the City of London.  This area industrialization via classical Marshallian districts saw wealth and prosperity being generated on a scale unseen until then in human history, however, the benifits were allocated in a haphazard fashion and many of the lower strata of the population did suffer the effects of pollution, such as poor air quality due to coal combustion products, unsafe working environments leading to deaths and maiming on the job, such that 40\% of the volunteers for the Boer War (1899-1902) were deemed medically unfit for service\footnote{Incidentally, this was a major impetus for the turn of the century public health movement, and also was a foundational plank in David Lloyd George's push for post-WWI public health policy\cite{hall2002cities}} \cite{hall2002cities}.  


%

%Firm  
%Too big to govern effectively
%	Banks - 
%Diseconomies of scale
%	Finance - waste of resources/locus for rent extraction
%	%	Q: You’re concerned that the financial sector is too big. Why?
%	%	A; The job of finance is to provide capital to companies. We do it to the tune of $250 billion a year in IPOs and secondary offerings. What else do we do? We encourage investors to trade about $32 trillion a year. So the way I calculate it, 99% of what we do in this industry is people trading with one another, with a gain only to the middleman. It’s a waste of resource
%
%	Competition between firms
%		Margin
%			Increasing completion should reduce profitability - all other assumptions being held 
%		Too Dominant
%		Labour
%			Too much competition for skilled labour may raise wages above profitabilty vs labour in another location
%	
%-Negative Network Effects
%	Congestion
%		Overload
%			Chicago Rail for logistics operations
%			Traffic Jam at rush hour
%	Pollution
%		Concentration of pollution 
%			Non-linear effects
%			

\section{Question 3}
\begin{quotation}
	\textbf{This question has the following parts: (1) Describe, explain and discuss ``Spatialization'' based on Fabrikant, Skupin et al.; (2) Next, provide a hypothetical for now, but as good-as-you-can-make-it, example of Spatialization using	some of the data for your study. Which of your data variables (pp. 14-15 in your Proposal) can be used? Pay attention to the X and Y axis, ensuring they make sense and have a clear label. State what these labels are; (3) What Distance Modeling could be applied to your example? (Cf. Chapter 6 on Distance Relationships by Chrisman); (4) Given `Investment DistanceSheds' generated in (3) in non-spatial `financial spaces', what Surface Modeling can be applied? (Cf. Chapter 8 on Spatial Analysis using Continuous Fields by Burrough et al); (5) What results can be expected from using this type of approach? Comment on this three-part `Surface Analysis of Distance Models derived from Spatialization of Investment Data' approach.}
\end{quotation}

\subsection{}
\subsection{}
\subsection{}
\subsection{}
\subsection{}


	\pagebreak
	\bibliographystyle{chicago}
	\bibliography{Mybibliography}
\end{document}