\documentclass[12pt,letterpaper,notitlepage,onecolumn,final,openbib]{article}
\usepackage[latin1]{inputenc}
\usepackage{amsmath}
\usepackage{amsfonts}
\usepackage{amssymb}
\usepackage[T1]{fontenc}
\usepackage{graphicx}
\usepackage{dcolumn}
\usepackage{rotating}
\usepackage{setspace}
\usepackage{epigraph}
\usepackage{url}
\usepackage{chicago}
\usepackage{booktabs}
\author{Martin Lefebvre}
\title{Comps}
\begin{document}
	
\section{Question 1}
\begin{quotation}
\textbf{It is commonly assumed that agglomeration economies generate benefit to: i) the firms within a cluster and ii) the communities that 'house' the agglomeration. In both cases, provide detail on the nature of these advantages.}
\end{quotation}
\section{Question 2}
\begin{quotation}
\textbf{Discuss possible conventional GIS modeling of some of your data variables (pp. 14-15 in your Proposal). This may include Overlay Combinations, Rules in Distance Relationships, Spatial Analysis using Continuous Fields and other functions, normally as part of GIS procedures (scripts). Allude to both (quantitative) Analysis and (qualitative) Visualization results. Specify the spatial scope for these models, e.g., within cities, regions (GTA, SW Ontario), or at coarse scales (Canada, USA, North America).}
\end{quotation}

Since 

Heat maps are a useful for discerning point patterns at various scales from continental (English North America) to regional (US Northeast\footnote{})neighbourhoods (Downtown GTA, Manhattan, ).   





\section{Question 3}
\begin{quotation}
\textbf{Discuss the concept of spatial autocorrelation. Explain and critically evaluate that concept in the context of the spatial patterns of economic activities.}
\end{quotation}

Spatial autocorrelation is at it's heart the application of Tobler's first law of geography: ``everything is related to everything else, but near things are more related than distant things.''\cite[page  236]{toblera1970}.  In other words, this concept test using probability on whether the characteristics of location A are similar (or dissimilar) to location B via distance lags.   Significant results implies that spatial processes are at work.  Furthermore, this underlying spatial pattern renders many classical statstics unreliable, since it violates the assumption of independence between observations. 

There are four widely used measures of spatial autocorrelation: Joint count analysis, Moran's I, Geary's C and Getis-Ord g statistic (also known in the literature as Hotspot Analysis or High/Low Clustering) as well as Local Moran's I.  

The simplest test for spatial autocorrelation is joint count analysis \cite{LeeWong2001}. This technique measures the strength of association of a phenomenon with regards to presence or absence of a binary attribute in adjacent polygons (ie sum of 1-1, 0-1, 0-0 interactions) vs an expected count that assumes the distribution of polygon counts is randomly distributed. As with many spatial techniques, this measure is sensitive to the border effect since unnecessarily large survey areas or unessessarily small polygons (using dissemination blocks for Canada-wide data location of Tim Hortons franchise locations) can artificially increase the effect of clustering.  The reverse is also true, with too large a spatial aggregation reducing the power of the test to irrelevance (ie. using presence in forward sortation areas polygons).   

While this method is simple (conceptually and mathematically), it's simplicity can make it a coarse instrument in some respects.  For instance, the binary attribute (presence or absence), does not measure the strength of association, therefore a polygon with 1 instance is weighted the same as a polygon with 10 instances.  In order to calculate this level of interaction, a more sophisticated measure will be needed, such as Moran's I statistic.  

Moran's I, is more sophisticated than joint-count analysis, by allowing the use of points and  polygons as well as the ability to use any continuous variable rather than binary data.  The output of Moran's I measures the degree of correlation between the variable $\chi$ and the spatial lag of $\chi$ by averaging the value of $\chi$ in neighbouring regions as determined by the spatial weights matrix. 

Furthermore, the spatial weights table is more customizable to the problem at hand than in Joint-Count Analysis.  Most of the decision process in choosing the distance method for populating the spatial weights table will be influence by the literature.  From the most naive (least subject knowledge/fewest assumptions), the spatial weights table can describe a basic agency matrix (1 or 0) using the rook's or queen's case when dealing with polygons, or in the case of point data, the nearness of the point/polygon to it's neighbours ( $wij = 1/dij$ where the matrix is calculated as 1 over the distance of between points) for $n$-number of neighbours or buffer of n-distance from each point/centroid of polygon.  Furthermore, the adjacency matrix can also be row standardized when there is an unequal number of neighbours, or have a distance decay function such as exponential decay.   

With regards to spatial pattern of economic activity, Moran's I can be used to test whether certain industries cluster or reppel each other.  For example, \citeN{GreenOLef2014} shows that institutional investors located on the island of Manhattan show clustering behaviour by co-locating more than what would be expected by random chance.  However, a more sophisticated treatment of Moran's I by mesuring the association between the weighted return on capital metric rather than simple presence, and found that metric was randomly distributed.  This makes conceptual sense, since it is unlikely that a potentially footloose industry (very few sunk costs with regards to location when compared to heavy industry)  investors could profit by taking advantage of spatial location alone on an ongoing period, especially when the dominant investing paradigm indicates that it is impossible to beat the market on a long-term basis \cite{Malkiel2005}.  

More broadly, while Moran's I can be used to test for the presence of spatial autocorrelation of an industry within a certain geographical area, it's can't pinpoint local clusters.  This is where the localized variant of Moran's I, also known as Local Indicators of Spatial Association (LISA) should be used \cite{anselin1995local}.  This variant of Moran's I simultaneously calculates the degree of spatial autocorrelation between one point/polygon to all neighbouring points/polygon in order to identify each point/polygon as belonging to one of 4 states (High/High, Low/Low (Similarity to neighbours), High/Low and Low/High(dissimilar to neighbours/possible outliers) as well as indicating the absence of significant relationships \cite{Anselin2003}.  With regards to using LISA with economic data, one could do worse than using this technique to find clusters of economic agglomeration.  For example, if analyzing the spatial autocorrelation of firm headquarters, cluster such as Toronto's Financial District would likely show up as a High-High district, whereas nearby Regent's Park would likely be a Low-High district (near Downtown, but relatively few/non-existent amount of headquarters compared to it's environs) . 



	\pagebreak
	\bibliographystyle{chicago}
	\bibliography{Mybibliography}
\end{document}