\documentclass[12pt,letterpaper,notitlepage,onecolumn,final,openbib]{article}
\usepackage[latin1]{inputenc}
\usepackage{amsmath}
\usepackage{amsfonts}
\usepackage{amssymb}
\usepackage[T1]{fontenc}
\usepackage{graphicx}
\usepackage{dcolumn}
\usepackage{rotating}
\usepackage{setspace}
\usepackage{epigraph}
\usepackage{url}
\usepackage{chicago}
\usepackage{booktabs}
\author{Martin Lefebvre}
\title{Comps}
\begin{document}
	
\section{Question 1}

\textbf{It is commonly assumed that agglomeration economies generate benefit to: i) the firms within a cluster and ii) the communities that 'house' the agglomeration. In both cases, provide detail on the nature of these advantages.}

\section{Question 2}

\textbf{Discuss possible conventional GIS modeling of some of your data variables (pp. 14-15 in your Proposal). This may include Overlay Combinations, Rules in Distance Relationships, Spatial Analysis using Continuous Fields and other functions, normally as part of GIS procedures (scripts). Allude to both (quantitative) Analysis and (qualitative) Visualization results. Specify the spatial scope for these models, e.g., within cities, regions (GTA, SW Ontario), or at coarse scales (Canada, USA, North America).}

\section{Question 3}
\textbf{Discuss the concept of spatial autocorrelation. Explain and critically evaluate that concept in the context of the spatial patterns of economic activities.}



\section{Bibliography}	
	\pagebreak
	\bibliographystyle{chicago}
	\bibliography{Mybibliography}
\end{document}